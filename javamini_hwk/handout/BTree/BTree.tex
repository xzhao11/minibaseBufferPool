
\documentstyle[psfig]{article}

\include{psfig}

\addtolength{\oddsidemargin}{-0.75in}
\setlength{\textwidth}{6.25in}


\font\bigheadfont=cmss12 at 12truept
\font\medheadfont=cmss10 at 10truept
\font\smallheadfont=cmssq8 at 8truept

\def\uwhead{{
\bigheadfont\centerline{UNIVERSITY OF WISCONSIN--MADISON}
\medheadfont\baselineskip=.43cm\centerline{COMPUTER SCIENCES DEPARTMENT}
\vskip.15cm
\centerline{CS 564:  DATABASE MANAGEMENT SYSTEMS}
\bigheadfont\baselineskip=1.2em \vskip.25cm
\centerline{Assignment 3:  B+ Trees}
\centerline{Due: September 27, 1996}
\medheadfont \vskip .2cm
\centerline{Instructors:  Jeff Naughton and Raghu Ramakrishnan}
\vskip .2cm}}

\def\endash{{\the\textfont0--}} %% Need these because PostScript fonts do not
\def\emdash{{\the\textfont0---}}%% make these dashes join together.

\renewcommand{\baselinestretch}{1.2}

\begin{document}

\uwhead

\section{Introduction}

In this assignment, you will implement a
B+ tree in which leaf level pages contain entries of the form
$\langle${\em key, rid of a data record}$\rangle$ (Alternative 2
for data entries, in terms of the textbook.) 
You must implement the full search and insert
algorithms as discussed in class.  In particular, your insert routine
must be capable of dealing with overflows (at any level of the tree)
by splitting pages; as per the algorithm discussed in class, you
will not consider re-distribution.  For this assignment, you can
deal with deletes by simply marking the corresponding leaf entry
as `deleted'; you do not have to implement merging.  

You will be given, among all other necessary packages and classes, 
HFPage and BTSortedPage.  BTSortedPage is derived from HFPage, and 
it augments  the
\verb+insertRecord+  method  of HFPage by   storing records on the
HFPage in sorted order by a specified {\em key} value.
The key value must be included as the initial part of each
record, to enable easy comparison of the key value of a new
record with the key values of existing records on a page.
The documentation available in the java code documentation is  
sufficient to understand what operation each function performs.

You need to implement two page-level classes,
{\bf BTIndexPage} and {\bf BTLeafPage}, both of which extends
BTSortedPage.
These page classes are used to build the B+ tree index;
you will write code to create, destroy, open and close a B+ tree index, and
to open scans that return all data entries (from the leaf pages)
which satisfy some range selection on the keys.

{\em  You will carry out this assignment in teams of
two, continuing with the same partner as for the previous assignment.
If you want to change partners, you must let us know first.}

\section{Getting Started}
\label{sec:start}

Please copy all the files from $\sim$cs564-1/fall96/proj3/src
into your own local directory. The files are:

\begin {itemize}
\item
{\em Makefile:}
A sample Makefile for you to compile your project.  
You will have to set up any dependencies by editing this file.
You can also design your own Makefile.  Whatever you do, please
make sure that you use the classpath provided in the original
Makefile.

\item
{\em Some compiled classes, BT.class, BTSortedPage.class, etc. }
The java documentations of those classes can be found at
http://path to the file.  Read the classes and methods descriptions 
carefully, for it will help you understand the program structure.

\item
{\em IndexFile.java, IndexFileScan.java}.
You will probably have to modify these two files depending on
how you decide to handle your exceptions.
\end{itemize}

You can find other useful utilities in the java documentation.

\section{Design Overview}

You should begin by (re-)reading the chapter
{\em Tree Structured Indexing} of the textbook to get an overview of 
the B+ tree layer.  There is also information about the B+ tree layer
in the HTML documentation.

\subsection{A Note on Keys for this Assignment}

You should note that key values are passed to functions using \verb+KeyClass+
objects (an abstract class).  
The  contents of a key should be interpreted using the
\verb+AttrType+ variable.  
The key can be either a string(attrString) or an integer(attrInteger), 
as per the definition of AttrType in {\em global} package.  We just 
implement these two kinds of keys in this assignment. 
If the key is a string, its value is stored in a \verb+StringKey+ class
which extends the \verb+KeyClass+ .  Likewise, if the key is an integer,
its value is stored in a \verb+IntegerKey+ class that also extends
the \verb+KeyClass+ . 

Although using the above structure, keys can be of (the more general 
enumerated) type AttrType, you can return an error message if the
keys are not of type attrString or attrInteger.  

The BTSortedPage class, which augments the
\verb+insertRecord+  method  of HFPage by storing 
records on a page in sorted order according to a specified {\em key} value,
assumes that the key value is included as the initial part of each
record, to enable easy comparison of the key value of a new
record with the key values of existing records on a page.

\subsection{B+ Tree Page-Level Classes}

These classes are summarized in Figure 1. 
Note again that you must NOT add any private data members  to  {\em 
BTIndexPage}  or  {\em  BTLeafPage}.  

\begin{figure}
\centerline{\psfig{figure=btpages.eps,height=2in}}
\caption{The C++ Classes used for the B+Tree pages}\label{f1:btpages}
\end{figure}

\begin {itemize}
\item
{\em HFPage:}
This is the base class, you can look at {\em heap} package to get more details.
\item
{\em BTSortedPage:}
This  class is  derived from the class HFPage.  Its only
function is to  maintain records on  a HFPage in  a sorted order.
Only the slot directory is re-arranged.  The data records remain in the
same positions on the page.  This exploits the fact that the rids
of index entries are not important: index entries (unlike data records)
are never `pointed to' directly, and are only accessed by searching the
index page.
\item
{\em BTIndexPage:}
This class is derived from  BTSortedPage. It inserts records  of
the type $\langle${\em key, pageNo}$\rangle$ on the  BTSortedPage. The records 
are sorted by the key.
\item
{\em BTLeafPage:}
This class is derived from BTSortedPage.  It inserts records of
the type $\langle${\em key, dataRid}$\rangle$ on the  BTSortedPage. 
{\em dataRid}
is the rid of the data record. The records are sorted by the key.
Further, leaf pages must be maintained in a doubly-linked list.
\end {itemize}

\subsection{Other B+ Tree Classes}

\begin{figure}
\begin{center}
\begin{picture}(0,0)%
\special{psfile=btreeone.pstex}%
\end{picture}%
\setlength{\unitlength}{0.012500in}%
%
\begingroup\makeatletter\ifx\SetFigFont\undefined
% extract first six characters in \fmtname
\def\x#1#2#3#4#5#6#7\relax{\def\x{#1#2#3#4#5#6}}%
\expandafter\x\fmtname xxxxxx\relax \def\y{splain}%
\ifx\x\y   % LaTeX or SliTeX?
\gdef\SetFigFont#1#2#3{%
  \ifnum #1<17\tiny\else \ifnum #1<20\small\else
  \ifnum #1<24\normalsize\else \ifnum #1<29\large\else
  \ifnum #1<34\Large\else \ifnum #1<41\LARGE\else
     \huge\fi\fi\fi\fi\fi\fi
  \csname #3\endcsname}%
\else
\gdef\SetFigFont#1#2#3{\begingroup
  \count@#1\relax \ifnum 25<\count@\count@25\fi
  \def\x{\endgroup\@setsize\SetFigFont{#2pt}}%
  \expandafter\x
    \csname \romannumeral\the\count@ pt\expandafter\endcsname
    \csname @\romannumeral\the\count@ pt\endcsname
  \csname #3\endcsname}%
\fi
\fi\endgroup
\begin{picture}(285,135)(55,640)
\end{picture}

\end{center}
\caption{Layout of a BTree with one BTLeafPage\label{fig:btreeone}}
\end{figure}

\begin{figure}
\begin{center}
\begin{picture}(0,0)%
\special{psfile=btreetwo.pstex}%
\end{picture}%
\setlength{\unitlength}{0.012500in}%
%
\begingroup\makeatletter\ifx\SetFigFont\undefined
% extract first six characters in \fmtname
\def\x#1#2#3#4#5#6#7\relax{\def\x{#1#2#3#4#5#6}}%
\expandafter\x\fmtname xxxxxx\relax \def\y{splain}%
\ifx\x\y   % LaTeX or SliTeX?
\gdef\SetFigFont#1#2#3{%
  \ifnum #1<17\tiny\else \ifnum #1<20\small\else
  \ifnum #1<24\normalsize\else \ifnum #1<29\large\else
  \ifnum #1<34\Large\else \ifnum #1<41\LARGE\else
     \huge\fi\fi\fi\fi\fi\fi
  \csname #3\endcsname}%
\else
\gdef\SetFigFont#1#2#3{\begingroup
  \count@#1\relax \ifnum 25<\count@\count@25\fi
  \def\x{\endgroup\@setsize\SetFigFont{#2pt}}%
  \expandafter\x
    \csname \romannumeral\the\count@ pt\expandafter\endcsname
    \csname @\romannumeral\the\count@ pt\endcsname
  \csname #3\endcsname}%
\fi
\fi\endgroup
\begin{picture}(425,325)(50,450)
\end{picture}

\end{center}
\caption{Layout of a BTree with more than one BTLeafPage\label{fig:btreetwo}}
\end{figure}

We will assume here that everyone understands the concept of B+ trees, and the
basic algorithms, and concentrate on explaining the design of the
C++ classes that you will implement.

A BTreeFile will contain a header page and a number of BTIndexPages and
BTLeafPages.  The header page is used to hold information about the
tree as a whole, such as the page id of the root page, the type of
the search key, the length of the key field(s) ( which has a fexed maximum 
size in this assignment), etc.
When a B+ tree index is opened, you should read the header page first,
and keep it pinned until the file is closed.
Given the name of the B+ tree index file, how can you locate the header
page?  The DB class (in {\em diskmgr} package) has a method
\begin{verbatim}
public void add_file_entry(String fname, PageId start_page_num); 
\end{verbatim}
that lets you register this information when a file fname is created.
There are methods for deleting and reading these `file entries'
($\langle$file name, header page$\rangle$ pairs) as well,
which can be used when the file is destroyed or opened.
The header page contains the page id of the root of the tree, and
every other page in the tree is accessed through the root page.

Figure 2 shows what a BTreeFile with only one
BTLeafPage looks like; the single leaf page is also the root.  
Note that there {\em is no} BTIndexPage in this
case.  Figure 3 shows a tree with a few
BTLeafPages, and this can easily be extended to contain multiple
levels of BTIndexPages as well.  

\subsubsection{IndexFile and IndexFileScan}

A BTree is one particular type of index.  There are other types, for example
a Hash index.  However, all index types have some
basic functionality in common.  We've taken this basic index functionality
and created a {\em abstract class} called IndexFile.  You won't implement
any methods for IndexFile.  However, any class derived from an IndexFile 
should support {\tt ~IndexFile()}, {\tt Delete()}, and {\tt
insert()}.  And you might have to modified the \verb+throws+ 
clause for the {\tt Delete()} and {\tt insert()} in IndexFile accordingly
depending on how you should decide to handle the exceptions.

Likewise, an IndexFileScan is a abstract class
that contains the basic functionality all index file scans should support.

\subsubsection{BTreeFile}

The main class to be implemented for this assignment is BTreeFile.  
BTreeFile is a derived class of the IndexFile class, which means 
a BTreeFile is a
kind of IndexFile.  However, since IndexFile is only an abstract class
{\em all} of the methods associated with IndexFile must be implemented
for BTreeFile.  

The methods to be implemented are described under the {\em btree}
package of the given java documenation.  Specifically, you will be
responsible to provide code for the following classes:

\begin{itemize}
\item BTreeFile class\\

\begin{itemize}
\item constructor \\
There are two constructors for BTreeFile: one that will only
open an index, and the other that will create a new index on
disk, with a given type and key size.  You can pass 0 for the
fourth parameter in the second constructor of BTreeFile because
it is not within the scope of this assignment.
If a page overflows (i.e., no space for the new entry), 
you should split the page. 
You may have to insert additional entries of the form
$\langle$key, id of child page$\rangle$ into the higher level
index pages as part of a split.
Note that this could recursively go all the way up to the root, possibly
resulting in a split of the root node of the B+ tree. \\

\item \verb+insert+ method \\
If a page overflows (i.e., no space for the new entry),
you should split the page.
You may have to insert additional entries of the form
$\langle$key, id of child page$\rangle$ into the higher level
index pages as part of a split.
Note that this could recursively go all the way up to the root, possibly
resulting in a split of the root node of the B+ tree.

\item \verb+Delete+ method \\
The Delete method simply
removes the entry from the appropriate BTLeafPage.  You are not required
to implement redistribution or
page merging when the number of entries falls below threshold.
\end{itemize}

\item BTIndexPage class \\
See the explanation in previous sections.

\item BTLeafPage class \\
See the explanation in previous sections.
\end{itemize}

Note that the java documentation provided to you for methods in
\verb+BTreeFile+,  
\verb+BTIndexPage+, and 
\verb+BTLeafPage+,  
do not specify whether they throw any exceptions; or rather,
what exceptions that they throw.  You should follow the error
protocol defined as before to implement your methods with
reasonable exceptions thrown.

\subsubsection{BTreeFileScan}

Finally, you will implement scans that return data entries from
the leaf pages of the tree.
You should create the scan
through a member of BTreeFile, so that you can report an error if a BTreeFile
is closed before a scan is completed.  

Note that BTreeFileScans should support several kinds
of range selections.  These
ranges are described in java documentation for {\tt BTreeFile} class.

\section{A Note on the Due Date}

This project should take you about two weeks to finish.  Unfortunately,
that would make this due around October 30, which is the date of our
midterm exam.  Since your instructors are such nice people, they have
decided not to have you finishing your assignment at the same time
you are preparing for the midterm.

However, we would strongly urge you to begin the assignment as if
it were due in two weeks.  Then you can take some time out to prepare
for the midterm, and will be in good shape to finish in the week
following that.  If you wait until after the midterm to start, you
will be unlikely to finish.

 
\section{Extra Credit}

{\em Extra credit of up to 15 percentage points is available.
However, the maximum number of points that you can score on this
assignment is 110.  (So the 15 extra points could be used partially to offset
points you lose elsewhere on the assignment.)}
The main motivation
for trying these additional challenges should be the opportunity to write
more complete software and understand some of the finer points, rather than to
score more points.  {\em Do not start on this until you have completed
the basic assignment!}

The tasks are listed below with the extra percentage points they
carry in parentheses:

\begin{enumerate}
\item [(1)]
{\em 5 points:}  Support duplicate records by allowing records with the same key
to exist on more than one page.  (You should not use any overflow pages!)
\item [(2)]
{\em 10 points:}  Implement node redistribution and merging during deletion.
\end{enumerate}
\end{document}


