
\documentstyle[psfig]{article}

\include{psfig}

\addtolength{\oddsidemargin}{-0.75in}
\setlength{\textwidth}{6.25in}


\font\bigheadfont=cmss12 at 12truept
\font\medheadfont=cmss10 at 10truept
\font\smallheadfont=cmssq8 at 8truept

\def\uwhead{{
\bigheadfont\centerline{UNIVERSITY OF WISCONSIN--MADISON}
\medheadfont\baselineskip=.43cm\centerline{COMPUTER SCIENCES DEPARTMENT}
\vskip.15cm
\centerline{CS 564:  DATABASE MANAGEMENT SYSTEMS}
\bigheadfont\baselineskip=1.2em \vskip.25cm
\centerline{Assignment 2:  Heap Files}
\centerline{Due: September 27, 1996}
\medheadfont \vskip .2cm
\centerline{Instructors:  Jeff Naughton and Raghu Ramakrishnan}
\vskip .2cm}}

\def\endash{{\the\textfont0--}} %% Need these because PostScript fonts do not
\def\emdash{{\the\textfont0---}}%% make these dashes join together.

\renewcommand{\baselinestretch}{1.2}

\begin{document}

\uwhead

\section{Introduction}

Welcome to Minibase, a new version of the Minirel system that has been
used in the CS564 course project for many years.   Minibase is a small
relational DBMS, structured  into several layers.  In this assignment,
you will  implement the Heap file layer.   You will  be given the documentation
for the lower layers (Buffer Manager and Disk  Space Manager), as well
as the documents for managing records on a Heap file page.  You can
find the package index for the above at \\

http://URL to the package...

{\em You will carry out this assignment, and subsequent ones, in teams
of two.  Please choose a partner as soon  as possible, or send mail to
michaell so that we can find a partner for you.}

This assignment has three parts.  You have to do the following:
\begin{enumerate}
\item
Familiarize  yourself with the Heap file, HFPage, Buffer Manager and
Disk Space Manager interfaces.
\item
Implement the Heap file class.  You can ignore concurrency control and
recovery issues,   and  concentrate   on implementing  the   interface
routines.  You should deal with free space intelligently, using either
a  linked  list or  page  directory to   identify pages  with room for
records.  When  a record is deleted,  you must update your information
about available  free space, and when a  record is inserted,  you must
try to use free space on existing pages before allocating a new page.
\item
Run the tests provided by the TA.
\end{enumerate}

\subsection{Available Documentation}

You should begin by reading the chapter  on Disks and Files (available
at MACC),  to get an overview of  the HF layer and buffer management.
This  material  was   also  covered in class.    In addition,  HTML
documentation   is available for  Minibase,  which  you can read using
Netscape.  There is a link to the Minibase home page in the CS564 home
page.

\section{Classes to Familiarize Yourself With First}

There are three main packages with which you should
familiarize yourself: {\em heap, bufmgr, diskmgr}.  Note that
part of the {\em heap} package contains implementation for
HFPage.  The java documentation of HFPage is provided to you.
A Heap file is seen as a collection of records.  Internally,
records are stored on a collection of HFPage objects.

\section{Compiling Your Code and Running the Tests}

Copy all the files  from {\em /p/course/cs564-1/assigns95/assign1} to
your own local assign1 directory  and study them carefully.  The files
are:

\begin {itemize}
\item
{\em Makefile:}  A sample Makefile  for  you to compile  your project.
Study this carefully before you make changes.  Whatever you do, make
sure that you do not modify the classpath which links your code to
the packages you may need.

\item
Again, the java documentation for packpage {\em bufmgr, diskmgr} and
{\em heap} are online. \\
{\em Note that you DO NOT have to throw the same exceptions as}
{\em documented for the heap package.  However, for testing}
{\em purpose, we DO ask you to name one of your exceptions}
{\em InvalidUpdateException to signal any illegal operations}
{\em on the record.  In other error situations, you should} 
{\em throw exceptions as you see fit following the error protocol}
{\em introduced in the Buffer Manager Assignment.}

\end{itemize}

\section{What to Turn In, and When}

You should turn  in copies of  your code  together with  copies of the
output produced   by running  the  tests  provided  by the   TAs.  The
assignment is due at 5PM on September 27th.  The solution will be made
public after that time,  and solutions turned in  after that time will
not receive any credit.  So be sure to turn in whatever you have, even
if it is not working fully, at that time.

{\em I emphasize  that late submissions  will not receive  any credit.
Computers\emdash     and life!\emdash  being   what   they are, expect
everything  to  take  longer   than  you   expect, even    taking this
expectation into account.  So start  early, and plan on getting things
done well before the due  date.  Nothing short  of a nuclear explosion
(in the CS building, not the South Pacific) constitutes a valid reason
for an extension.

\end{document}




