\documentstyle[psfig]{article}

\include{psfig}

\addtolength{\oddsidemargin}{-0.75in}
\setlength{\textwidth}{6.25in}


\font\bigheadfont=cmss12 at 12truept
\font\medheadfont=cmss10 at 10truept
\font\smallheadfont=cmssq8 at 8truept

\def\uwhead{{
\bigheadfont\centerline{UNIVERSITY OF WISCONSIN--MADISON}
\medheadfont\baselineskip=.43cm\centerline{COMPUTER SCIENCES DEPARTMENT}
\vskip.15cm
\centerline{CS 564:  DATABASE MANAGEMENT SYSTEMS}
\bigheadfont\baselineskip=1.2em \vskip.25cm
\centerline{Assignment 4:  Sort-Merge Join}
\centerline{Due: September 27, 1996}
\medheadfont \vskip .2cm
\centerline{Instructors:  Jeff Naughton and Raghu Ramakrishnan}
\vskip .2cm}}

\def\endash{{\the\textfont0--}} %% Need these because PostScript fonts do not
\def\emdash{{\the\textfont0---}}%% make these dashes join together.

\renewcommand{\baselinestretch}{1.2}

\begin{document}

\uwhead


\section{Introduction}

In this assignment,
you will implement the sort-merge join algorithm.
{\em You will carry out this assignment in teams
of two with the same partner as for the previous assignments.}

\section{Available Documentation}

You should begin by reading the chapter on 
{\em Implementation of Relational Operations},
in particular, the section on {\em Sort-Merge Join}.

The package that contains SortMerge is {\em iterator}.
You should also read the java documentation carefully for
the package.  You may find them at : http://path.
Although there are many classes under the {\em iterator} class,
you most likely only need a handful of them.  However, you are
encouraged to go over the rest of the documentation to gain
an understanding of how things are being done.  It will also  
help you to get started with your implementation.

\section{What You Have to Implement}

The sortMerge constructor joins two relations R and S,
represented by the \verb+Iterator+ object {\em am1} and {\em am2}, 
respectively,
using the sort-merge join algorithm. \\

Read the \verb+TupleUtils+ class carefully.  You will need to call
the \verb+setup_op_tuple+ method to set up joined tuple.
You will also need to call the compare functions to compare the
join columns of two tuples. \\ 
You will call the Sort constructor to sort a relation (represented
by \verb+Iterator+ class).\\
You will also have to implement the \verb+get_next()+ method that
returns the next joined tuple. \\
Note that all the relations are not saved in a heapfile or written
out to a heapfile.  Instead, you should use the functions provided
by \verb+IoBuf+ class to operate tuples through a buffer.

\section{Where to Find Makefiles, Code, etc.}

Copy all the files from $\sim$cs564-1/spring96/assign5/src
into your own local directory.
The files are:
\begin{itemize}
\item
	{\em Makefile}: A sample makefile to compile your project.
\item
	{\em *.class} : A compiled version of functions provided.
\item
	{\em tests} : package to test and run the SortMerge program.
\end{itemize}
Again, you can find the interfaces for all necessary interfaces
at http://path

\section{What to Turn In, and When}

You should turn  in copies of  your code  together with  copies of the
output produced   by running  the  tests  provided  by the   TAs.  The
assignment is due at 5PM on April 17.  The solution will be made
public after that time,  and solutions turned in  after that time will
not receive any credit.  So be sure to turn in whatever you have, even
if it is not working fully, at that time.

{\em I emphasize  that late submissions  will not receive  any credit.
Computers\emdash     and life!\emdash  being   what   they are, expect
everything  to  take  longer   than  you   expect, even    taking this
expectation into account.  So start  early, and plan on getting things
done well before the due  date.  Nothing short  of a nuclear explosion
(in the CS building, not the South Pacific) constitutes a valid reason
for an extension.

\end{document}

