\documentstyle[psfig]{article}

\include{psfig}

\addtolength{\oddsidemargin}{-0.75in}
\setlength{\textwidth}{6.25in}


\font\bigheadfont=cmss12 at 12truept
\font\medheadfont=cmss10 at 10truept
\font\smallheadfont=cmssq8 at 8truept

\def\uwhead{{
\bigheadfont\centerline{UNIVERSITY OF WISCONSIN--MADISON}
\medheadfont\baselineskip=.43cm\centerline{COMPUTER SCIENCES DEPARTMENT}
\vskip.15cm
\centerline{CS 564:  DATABASE MANAGEMENT SYSTEMS}
\bigheadfont\baselineskip=1.2em \vskip.25cm
\centerline{Assignment 1:  Buffer Manager }
\centerline{Due: September 27, 1996}
\medheadfont \vskip .2cm
\centerline{Instructors:  Jeff Naughton and Raghu Ramakrishnan}
\vskip .2cm}}

\def\endash{{\the\textfont0--}} %% Need these because PostScript fonts do not
\def\emdash{{\the\textfont0---}}%% make these dashes join together.

\renewcommand{\baselinestretch}{1.2}

\begin{document}

\uwhead

\section{Introduction}

In this assignment,
you will  implement a simplified version of the Buffer Manager layer,
without support for concurrency control or recovery.   You will be given 
the code for the lower layer, the Disk  Space Manager.

{\em You will carry out this assignment, and subsequent ones, in teams
of two.  Please choose a partner as soon  as possible, or send mail to
xbao or cychan so that we can find a partner for you.}

You should begin by reading the chapter  on Disks and Files,  
to get an overview of buffer management.
This  material will also be covered in class.  In addition,  HTML
documentation is available for  Minibase,  which  you can read using
Netscape.  There is a link to the Minibase home page in the CS564 home
page.  In particular, you should read the description of the DB class,
which you will call extensively in this assignment.  The Java documentation
for the {\em diskmgr} package can be found at 
~javaminibase/javamini\_hwk/handout/BufMgr/javadoc/packages.html.
You should also read the code under diskmgr/ carefully to learn
how package is declared and how Exceptions are handled in Minibase.

\section{The Buffer Manager Interface}

The simplified Buffer Manager interface that you will 
implement in this assignment
allows a client (a higher level program that calls the Buffer Manager) to
allocate/de-allocate pages on disk, to bring a disk page into the buffer pool
and pin it, and to unpin a page in the buffer pool.  


The methods that you have to implement are described below:

\begin{verbatim}

public class BufMgr {

  /**
   * Create the BufMgr object.
   * Allocate pages (frames) for the buffer pool in main memory and
   * make the buffer manage aware that the replacement policy is
   * specified by replacerArg (i.e. Clock, LRU, MRU etc.).
   *
   * @param numbufs number of buffers in the buffer pool.
   * @param replacerArg name of the buffer replacement policy.
   */

  public BufMgr(int numbufs, String replacerArg) {};


  /** 
   * Pin a page.
   * First check if this page is already in the buffer pool.  
   * If it is, increment the pin_count and return a pointer to this 
   * page.  If the pin_count was 0 before the call, the page was a 
   * replacement candidate, but is no longer a candidate.
   * If the page is not in the pool, choose a frame (from the 
   * set of replacement candidates) to hold this page, read the 
   * page (using the appropriate method from {\em diskmgr} package) and pin it.
   * Also, must write out the old page in chosen frame if it is dirty 
   * before reading new page.  (You can assume that emptyPage==false for
   * this assignment.)
   *
   * @param Page_Id_in_a_DB page number in the minibase.
   * @param page the pointer poit to the page.
   * @param emptyPage true (empty page); false (non-empty page)
   */

  public void pinPage(PageId pin_pgid, Page page, boolean emptyPage) {};


  /**
   * Unpin a page specified by a pageId.
   * This method should be called with dirty==true if the client has
   * modified the page.  If so, this call should set the dirty bit 
   * for this frame.  Further, if pin_count>0, this method should 
   * decrement it. If pin_count=0 before this call, throw an exception
   * to report error.  (For testing purposes, we ask you to throw
   * an exception named PageUnpinnedException in case of error.)
   *
   * @param globalPageId_in_a_DB page number in the minibase.
   * @param dirty the dirty bit of the frame
   */

  public void unpinPage(PageId PageId_in_a_DB, boolean dirty) {};


  /** 
   * Allocate new pages.
   * Call DB object to allocate a run of new pages and 
   * find a frame in the buffer pool for the first page
   * and pin it. (This call allows a client of the Buffer Manager
   * to allocate pages on disk.) If buffer is full, i.e., you 
   * can't find a frame for the first page, ask DB to deallocate 
   * all these pages, and return null.
   *
   * @param firstpage the address of the first page.
   * @param howmany total number of allocated new pages.
   *
   * @return the first page id of the new pages.  null, if error.
   */

  public PageId newPage(Page firstpage, int howmany) {};

		
  /**
   * This method should be called to delete a page that is on disk.
   * This routine must call the method in diskmgr package to 
   * deallocate the page. 
   *
   * @param globalPageId the page number in the data base.
   */

  public void freePage(PageId globalPageId) {};


  /**
   * Used to flush a particular page of the buffer pool to disk.
   * This method calls the write_page method of the diskmgr package.
   *
   * @param pageid the page number in the database.
   */

  public void flushPage(PageId pageid) {};

};

\end{verbatim}


\section{Internal Design}

The {\em buffer pool} is a collection of {\em frames} (page-sized 
sequence of main memory bytes) that is managed by the Buffer Manager.
Conceptually, it should be stored as an array bufPool[numbuf] of 
Page objects.  However, due to the limitation of Java language, it
is not feasible to declare an array of Page objects and later on 
writing string (or other primitive values) to the defined Page.
To get around the problem, we have defined our Page as an array of 
bytes and deal with buffer pool at the byte array level.  Therefore, 
when implement your constructor for the BufMgr class, you should declare
the buffer pool as {\em bufpool[numbuf][page\_size]}.  Note that the
size of minibase pages is defined in the interface {\em GlobalConst} of
the {\em global} package.  Before jump into coding, please make sure
that you understand how Page object is defined and manipulated in Java
minibase.


In addition, you should maintain an array bufDescr[numbuf] of 
{\em descriptors}, one per frame.  Each descriptor is a record with 
the following fields:
\begin{quote}
{\em page number, pin\_count, dirtybit}.
\end{quote}
The {\em pin\_count} field is an
integer, {\em page number} is a PageId object, and {\em dirtybit}  
is a boolean. This describes the page that is stored in the corresponding frame.
A page is identified by
a {\em page number} that is
generated by the DB class when the page is allocated, and is unique over all
pages in the database. 
The PageId type is defined as an integer type in minirel.h.

A simple {\em hash table} should be used to figure out 
what frame a given disk page occupies.  
The hash table should be implemented (entirely in main memory)
by using an array of pointers to lists of
$\langle$ {\em page number, frame number} $\rangle$ pairs.  
The array is called the {\em directory} and each list of
pairs is called a {\em bucket}.  Given a {\em page number}, you
should apply a {\em hash function} to find the directory entry
pointing to the bucket that contains the frame number for this page, if the
page is in the buffer pool.  If you search the bucket and don't find
a pair containing this page number, the page is not in the pool.
If you find such a pair, it will tell you the frame in which the page
resides.  This is illustrated in Figure \ref{assign1fig}.

\begin{figure}
\centerline{\psfig{figure=assign1fig.eps}}
\caption{Hash Table}\label{assign1fig}
\end{figure}

The hash function
must distribute values in the domain of the search field
uniformly over the collection of buckets.
If we have HTSIZE buckets, numbered 0 through M-1, a hash
function $h$ of the form $h(value)~=~(a*value+b)~mod~HTSIZE$
works well in practice.  HTSIZE should be chosen to be a prime number.

When a page is requested the buffer manager should do the following:
\begin{enumerate}
\item
Check the buffer pool (by using the hash table) to see if it contains the requested page.
If the page is not in the pool, it should be brought in as follows:
\begin{enumerate}
\item
Choose a frame for replacement,
using the Clock replacement policy.
\item
If the frame chosen for replacement is dirty, 
{\em flush} it (i.e., write out the page that it contains to disk,
using the appropriate DB class method).
\item
Read the requested page (again, by calling the DB class) into the frame chosen for replacement; the 
{\em pin\_count} and {\em dirtybit} for the frame
should be initialized to 0 and FALSE, respectively.
\item
Delete the entry for the old page from the Buffer Manager's hash table and insert
an entry for the new page.
Also, update the entry for this frame in the bufDescr array to reflect these changes.
\end{enumerate}
\item
{\em Pin} the requested page by incrementing the {\em pin\_count} in the descriptor
for this frame.
and return a pointer to the page to the requestor.  
\end{enumerate}

To implement the Clock replacement policy, maintain a 
queue of frame numbers.
When a frame is to be chosen for replacement, you should pick the first frame 
in this list.
A frame number is added to the end of the queue when the {\em pin\_count} for
the frame is decremented to 0.
A frame number is removed from the queue if the {\em pin\_count} becomes
non-zero:  this could happen if there is a request for the page currently in
the frame!


\section{Where to Find Makefiles, Code, etc.}


Please copy all the files from $\sim$cs564-1/fall96/proj1/src
into your own local directory. The contents are:

\begin{itemize}
\item under {\em bufmgr} directory \\
        You should have all your code for {\em bufmgr} package implemented
        under this directory.  We provide a sample {\em Makefile} to compile 
        your project. You will have to set up any dependencies by editing 
        this file.
	You can also design your own Makefile.  Whatever you do, please
        make sure that the classpath is set to the one provided in
        the original Makefile.

\item under {\em diskmgr} direcotry \\
        You should study the code for {\em diskmgr} package carefully before
        you implement the {\em bufmgr} package.  The detailed description
        of the files are not included.  Please refer to the java documentation
        of the packages.

\item under {\em tests} directory \\
	{\em Makefile}, {\em TestDriver.java}, {\em BMTest.java}: 
	Buffer manager test driver program.  Note that you do have to
        use the Makefile to run your tests too.
\end{itemize}

You can find other useful information by reading the java documentation
on other packages, such as the {\em global} package and the 
{\em chainexception} package.

\section{Error Protocol}

Though the Throwable class in Java contains a snapshot of the execution 
stack of its thread at the time it was created and also a message
string that gives more information about the error (exception), we have
decided to maintain a copy of our own stack to have more control over
the error handling.

We provide the {\em chainexception} package to handle the Minibase 
exception stack.  Every exception created in your {\em bufmgr} package
should extend the ChainException class.  The exceptions are thrown
according to the following rule:

\begin{itemize}

\item Error caused by an exception caused by another layer: \\

\begin{verbatim}

  For example: (when try to pin page from diskmgr layer)

    try {
      SystemDefs.JavabaseBM.pinPage(pageId, page, true);
    }
    catch (Exception e) {
      throw new DiskMgrException(e, "DB.java: pinPage() failed");
    }


\end{verbatim}


\item Error not caused by exceptions of others, but needs to be
      acknowledged.

\begin{verbatim}

  For example: (when try to unpin a page that is not pinned)

        if (pin_count == 0) {
          throw new PageUnpinnedException (null, "BUFMGR: PAGE_NOT_PINNED.");
        }

\end{verbatim}
\end{itemize}


Basically, the ChainException class keeps track of all the exceptions 
thrown accross the different layers.  Any exceptions that you decide
to throw in your {\em bufmgr} package should extend the ChainException
class. \\

For testing purposes, we ask you to throw the following exceptions
in case of error (use the exact same name, please):
\begin{itemize}
\item BufferPoolExceededException \\
      Throw a BufferPoolExceededException when try to pin
      a page to the buffer pool with no unpinned frame left.

\item PagePinnedException \\
      Throw a PagePinnedException when try to free a page that
      is still pinned.
      
\item HashEntryNotFoundException \\
      Throw a HashEntryNotFoundException when page specified by
      PageId is not found in the buffer pool.
\end{itemize}

Feel free to throw other new exceptions as you see fit.  But make 
sure that you follow the error protocol when you catch an exception.  
Also, think carefully about what else you need to do in the 
{\em catch} phase. Sometimes you do need to unroll the previous 
operations when failure happens.

\section{What to Turn In, and When}

You should turn  in copies of your code together with  copies of the
output produced   by running  the  tests  provided  by the   TAs.  The
assignment is due at 5PM on September 27th.  The solution will be made
public after that time,  and solutions turned in  after that time will
not receive any credit.  So be sure to turn in whatever you have, even
if it is not working fully, at that time.

{\em I emphasize  that late submissions  will not receive  any credit.
Computers\emdash     and life!\emdash  being   what   they are, expect
everything  to  take  longer   than  you   expect, even    taking this
expectation into account.  So start  early, and plan on getting things
done well before the due  date.  Nothing short  of a nuclear explosion
(in the CS building, not the South Pacific) constitutes a valid reason
for an extension.
\end{document}



